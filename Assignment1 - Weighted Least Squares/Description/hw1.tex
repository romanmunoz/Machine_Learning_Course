\documentclass{llncs}

\usepackage[margin=1in]{geometry}

\usepackage{graphicx,color,comment,url} 

\usepackage{amsmath,amssymb}


\title{4033/5033: Assignment 1}
\author{Your Name}
\institute{}

\begin{document}

\maketitle 

\setlength\parindent{0pt} 
\setlength{\parskip}{10pt}

Due: Sep 23 (by 11:59pm)

\textbf{Problem 1}. 
Weighted least square is a technique to learn 
linear regression models which weighs instances during training. Its objective function is 
\begin{equation}
J(\beta) = {\sum}_{i=1}^{n} w_{i} 
\cdot (x_{i}^{T}\beta - y_{i})^2
\end{equation}
where $w_{i} \in \mathbb{R}$ is the weight for instance 
$x_{i} \in \mathbb{R}^{p}$. If a $w_i$ is large, then 
$\beta$ will focus on fitting $x_i$ and thus gain 
lower error on $x_i$ (in theory).  
After $\beta$ is learned, we can apply it 
to predict label for any instance $z$ by 
$z^T \beta$. 
Complete the following three tasks. 

\textit{Task 1}. Derive the matrix form of $J(\beta)$. 
In the result, you can use the following notations: 
$X$ is an $n$-by-$p$ matrix with $x_i^T$ on the $i_{th}$ 
row and $Y$ is an $n$-dimensional vector with 
$y_i$ being its $i_{th}$ element.\footnote{Tip: 
use the matrix form of $\sum_i w_i a_i^2$.}  

\textit{Task 2}. Derive an analytic solution of 
$\beta$ and present it in the matrix form. 

\textit{Task 3}. Implement your solution in Python 
from scratch. Test your implementation on the given 
Community Crime data set and report experimental 
results. Use the following experiment design. 

-- Randomly select 75\% data for training and use 
the rest 25\% data for testing. Repeat the random 
trial for 10 times and report the average 
testing error (RMSE). 

-- Let us define two groups in the community: 
(i) high crime rate group contains community instances 
whose labels are bigger than 0.8; (ii) low crime rate 
group contains community instances whose labels are 
no bigger than 0.8. After experimenting weighted least 
square, let us report its testing 
errors on different groups 
in Table \ref{tab1}. 
(We will use $w_h$ to denote weight for instances in 
the high crime rate group and $w_\ell$ to denote 
weight for instances in the low crime rate group.)

-- Report the average number of 
training instances in the two groups in Table 
\ref{tab2}. 

\begin{table}[]
    \centering
    \setlength{\tabcolsep}{5pt}
    \renewcommand{\arraystretch}{1.5}
    \begin{tabular}{l|c|c|c|c}
    \hline 
    \bf Choice of Weights  
    & \bf $w_{\ell} = 1, w_h = 1$ 
    & \bf $w_{\ell} = 1, w_h = 10$
    & \bf $w_{\ell} = 1, w_h = 50$ 
    & \bf $w_{\ell} = 1, w_h = 0.1$  \\ \hline 
Error on all testing instances & ... & ... 
& ... & ... \\
Error on high crime rate group & ... & ... 
& ... & ...\\
Error on low crime rate group & ... & ... 
& ... & ...\\ 
\hline 
\end{tabular}
\caption{Performance of Weighted Least Square}
\label{tab1}
\end{table}


\begin{table}[]
    \centering
    \setlength{\tabcolsep}{5pt}
    \renewcommand{\arraystretch}{1.5}
    \begin{tabular}{l|c}
    \hline 
    \# Instances in the High Crime Group 
    & ............  \\ \hline
    \# Instances in the Low Crime Group 
    & ............  \\ \hline 
\end{tabular}
\caption{Number of Instances in the Two Groups}
\label{tab2}
\end{table}

\newpage 

\textbf{Problem 2}. Implement Lasso plus 
coordinate descent (CD) and evaluate it on 
the Community Crime data set. Use the 
following experiment design. 

-- Use the first 75\% data for training 
and the rest 25\% for testing. 

-- Pick a proper regularization 
coefficient yourself so you can get 
a sparse model. (You can examine the 
model coefficients after it is learned.)

-- Fix the picked coefficient, report 
performance of your implemented algorithm 
in Figure \ref{fig1} and Figure \ref{fig2}. 

Figure \ref{fig1} should contain a curve of 
testing error versus the number of CD updates (y-axis is testing 
error and x-axis is number of CD updates). You 
should choose a proper range of x-axis so we 
can observe convergence of your testing error. 

\begin{figure}[h]
    \centering
    \includegraphics{}
    \caption{Testing Error versus CD Updates}
    \label{fig1}
\end{figure}

Figure \ref{fig2} should contain a curve of the number of non-zero elements in your model 
versus the number of CD updates (y-axis is number of non-zero elements, and x-axis is number of CD updates). The range of x-axis should be same 
as in Figure \ref{fig1}. 

\begin{figure}[h]
    \centering
    \includegraphics{}
    \caption{Number of Non-Zero Elements in 
    $\beta$ versus CD Updates}
    \label{fig2}
\end{figure}

\vfill

\underline{Submission Instruction}

Please submit three files to Canvas. 

(i) Submit a `hw1.pdf'. It should contain 
your answers to all the questions. 
For mathematical questions, you can 
write the answers on a paper, scan it and 
include it in the pdf file; or, you can 
also directly type the answers in Latex 
and compile them into pdf. For experimental 
questions, you need to draw the figures 
using Python and include them in the pdf file. 

(ii) Submit a `hw1\_WLS.py'. It should 
be the code of your implemented weighted 
least square. Please make sure we can 
directly run your code without changing 
any parts (expect directory of the data 
set maybe). 

(iii) Submit a `hw1\_Lasso.py'. It should 
be the code of your implemented Lasso. 
Please make sure we can directly run your 
code without changing any parts (expect 
directory of the data set maybe). 

\end{document}
