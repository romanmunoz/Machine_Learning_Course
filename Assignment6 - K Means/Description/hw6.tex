\documentclass{llncs}

\usepackage[margin=1in]{geometry}

\usepackage{graphicx,color,comment,url} 

\usepackage{amsmath,amssymb}


\title{4033/5033: Assignment 6}
\author{Your Name}
\institute{}

\begin{document}

\maketitle 

\setlength\parindent{0pt} 
\setlength{\parskip}{10pt}

Due: Nov 28 (by 11:59pm)

In this assignment, we play with K-means.\\  
Let $x_{1}, \ldots, x_n$ be a set of instances 
in $\mathbb{R}^p$ and $c_1, \ldots, c_K$ be a 
set of cluster centers in the same space.\\  
Let $\mathbb{I}_{ij}$ be a variable such that 
$\mathbb{I}_{ij} = 1$ if 
$x_i$ is assigned to cluster centered at $c_j$ 
and $\mathbb{I}_{ij} = 0$ otherwise.\\  

[1] In theory, we know K-means is equivalent 
to minimizing the following objective
\begin{equation}
\sum_{i=1}^n \sum_{j=1}^K \mathbb{I}_{ij} 
||x_i - c_j||^2.
\end{equation}
Specifically, its step that fixes clustering center 
and assign instances to their nearest centers is 
equivalent to fixing $c_j$ while optimizing $\mathbb{I}_{ij}$ 
in (1), and its step that fixes cluster assignment 
and updates cluster centers is equivalent to 
fixing $\mathbb{I}_{ij}$ while optimizing $c_j$ 
in (1). Based on this, answer the following question. 

Suppose we want to modify the K-means algorithm so it 
is equivalent to minimizing the following objective \begin{equation}
\sum_{i=1}^n \sum_{j=1}^K \mathbb{I}_{ij} 
w_{i} ||x_i - c_j||^2, 
\end{equation}
where $w_i$ is a weight of instance $x_i$. 
Explain your modified algorithm. In the answer, 
clearly state how each step in K-means is modified 
(if necessary) e.g., Step X is not changed, and 
Step Y is changed to ....\\  

[2] Implement standard K-means algorithm from scratch, 
evaluate it on the diabetes data set and visualize 
the clustering result in 2-dimensional space 
using the PCA technique. 

Specifically, you need to draw three figures. 
Figure 1 plots the data distribution with K = 2. 
Figure 2 plots the data distribution with K = 3. 
Figure 3 plots the data distribution with K = 5. 
(Pick proper color for each cluster yourself.) 

Also, report the random index and Davies–Bouldin index of 
each clustering result. You can use existing functions 
to evaluate these indices.
\url{https://scikit-learn.org/stable/modules/clustering.html#clustering-performance-evaluation}. 

Tip: we only run clustering on the features not 
on labels. The labels are only used for external 
evaluation. Practically, exclude the last column 
in `data' (which is label) when running clustering. 

\vfill 
 
\underline{Submission Instruction}

Please submit two files to Canvas. (Do not 
zip them. Upload them separately.) 

(i) All your mathematical and experimental results 
should be presented in a single pdf file named 
as `hw6.pdf'. 

(ii) A Python source code for the 
implementation of K-means named `hw6\_Kmeans.py'

\end{document}
