\documentclass{llncs}

\usepackage[margin=1in]{geometry}

\usepackage{graphicx,color,comment,url} 

\usepackage{amsmath,amssymb}


\title{4033/5033: Assignment 2}
\author{Your Name}
\institute{}

\begin{document}

\maketitle 

\setlength\parindent{0pt} 
\setlength{\parskip}{10pt}

Due: Sep 30 (by 11:59pm)

Previously, we studied a weighted 
least square (WLS) learning technique 
\begin{equation}
J(\beta) = {\sum}_{i=1}^{n} w_{i} 
\cdot (x_{i}^{T}\beta - y_{i})^2
\end{equation}
where $w_{i} \in \mathbb{R}$ is the weight for instance $x_{i} \in \mathbb{R}^{p}$. 


Please design some probabilistic assumptions 
and prove that, under those assumptions, the 
MLE estimation of your distribution parameter 
is the same as the solution to the above WLS problem. 

You need to 

-- Clearly list ALL the assumptions you made 

-- Clearly explain the meaning of every notation, 
especially if it is not commonly used in the lectures

-- Clearly explain the dimension of every 
matrix or vector e.g., $X \in \mathbb{R}^{n 
\times p}$  

-- Clearly elaborate the arguments to derive 
from MLE to WLS

-- Clearly point out which part of your derived 
results corresponds to the weight. (For example, 
in lecture we show the $\frac{\sigma^2}{t^2}$ 
in MAP estimate  corresponds to the regularization 
coefficient $\lambda$ in ridge regression.)  

\vfill

\underline{Submission Instruction}

Please submit a single pdf file `hw2.pdf' to Canvas. 

You can directly type the answers in Latex and compile 
them into pdf, or first hand-write the answers and 
scan them into pdf. In the latter case, please make 
sure your hand-writing is clear. If the grader does 
not understand the writing, you will lose points at 
that part. 

Parts of the answers will be graded based on hard criterion e.g., lose X points if an assumption is missing. The other parts will be graded based on subjective evaluation e.g., how much partial credit can be given to an argument that does not lead to the correct result or contains flaws. The subjective evaluation will not be open for negotiation. 

\end{document}
